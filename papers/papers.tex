\documentclass[acmsmall]{acmart}
%
% framed
%
\usepackage{framed}
%
% math indicator function
%
\usepackage{bbm}

%
% multirow multicolumn table
%
\usepackage{multirow}
%
% makecell
%
\usepackage{makecell}
%
% for lines in tables
%
\usepackage{hhline}
%
% added for accomodating long tables
%
\usepackage{longtable}
%
% diagbox in table
%
\usepackage{diagbox}
%
% make fonts of all tables smaller
%
% https://tex.stackexchange.com/questions/286733/change-content-font-in-tabular-environments
\makeatletter
\newcommand*\my@starttable[1][]{%
	  \@float{table}[#1]\footnotesize
		}
		\patchcmd{\table}{\@float{table}}{\my@starttable}{\PackageInfo{mysty}{Table environment patched successfully.}}{\PackageWarning{mysty}{Could not patch table environment.}}
\makeatother

\let\oldlongtable\longtable
\def\longtable{\small\oldlongtable}

%
% for program listing
%
\usepackage{listings}
% styles are in newstyles.tex

%
% strip on two columns
%
\usepackage{cuted}

%
% for each list numbering
%
\usepackage{enumitem}


%
% graphs
%
\usepackage{tikz}
\usetikzlibrary{fit,arrows,arrows.meta,shapes,shapes.geometric,positioning,calc,shadows,trees}

%
% subfigures
%
\usepackage[labelformat=simple]{subcaption}
\renewcommand\thesubfigure{(\alph{subfigure})}
%
% easy quotations
%
\usepackage[autostyle]{csquotes}

%
% code listing
%
\usepackage{listings}

%% Rights management information.  This information is sent to you
%% when you complete the rights form.  These commands have SAMPLE
%% values in them; it is your responsibility as an author to replace
%% the commands and values with those provided to you when you
%% complete the rights form.
\setcopyright{acmcopyright}
\copyrightyear{2018}
\acmYear{2021}
\acmDOI{xx.xxxx/xxyyzza.aabbcde}

%%
%% These commands are for a JOURNAL article.
\acmJournal{CSUR}
\acmVolume{Vol}
\acmNumber{Num}
\acmArticle{1}
\acmMonth{5}

%%
%% Submission ID.
%% Use this when submitting an article to a sponsored event. You'll
%% receive a unique submission ID from the organizers
%% of the event, and this ID should be used as the parameter to this command.
%%\acmSubmissionID{123-A56-BU3}

%%
%% The majority of ACM publications use numbered citations and
%% references.  The command \citestyle{authoryear} switches to the
%% "author year" style.
%%
%% If you are preparing content for an event
%% sponsored by ACM SIGGRAPH, you must use the "author year" style of
%% citations and references.
%% Uncommenting
%% the next command will enable that style.
%%\citestyle{acmauthoryear}

\let\Bbbk\relax    % must have this to use todo package; otherwise, a
\usepackage{todo}  % conflict, which results in a compilation error
\usepackage{xspace}

% math
	\newcommand{\cardinality}[1]{\left\vert{#1}\right\vert}
	\newcommand{\set}[1]{\mathbb{#1}}
	\newcommand{\vect}[1]{\boldsymbol{\mathbf{#1}}}
	\newcommand{\mtx}[1]{\mathbf{#1}}
	\newcommand{\dlist}[1]{\mathbf{#1}}
	\newcommand{\argmax}{\operatornamewithlimits{argmax}}
	\newcommand{\idemsg}[1]{\text{\tt \textless {#1}\textgreater}}
	%\newcommand{\idestate}[1]{\text{\sc \textless {#1}\textgreater}}
	\newcommand{\idestate}[1]{\text{\sc [{#1}]}}
	\newcommand{\seqprefix}{\sqsubseteq}

% programming
	\newcommand{\code}{\texttt}
	\newcommand{\codesf}{\textsf}
	\newcommand{\eqcode}[1]{\text{\tt #1}}
% staistics 
	\newcommand{\E}[1]{\operatorname{E}\left[#1\right]}
	\newcommand{\Cov}[1]{\operatorname{Cov}\left[#1\right]} 
	\newcommand{\Var}[1]{\operatorname{Var}\left[#1\right]}
% frequently used sets
	\newcommand{\N}{\mathbb N}
	\newcommand{\Q}{\mathbb Q}
	\newcommand{\R}{\mathbb R}

% formal languages
	\newcommand{\abstr}[1]{\text{\textless #1\textgreater\;}}
	\newcommand{\term}[1]{\text{\;#1\;}}
	\newcommand{\attr}[1]{\text{#1}}

	\newcommand{\pd}[1]{{#1}}

	\newcommand{\centercell}[1]{\multicolumn{1}{c}{#1}}

% digital library and search tool
	\newcounter{dlibc}
	\setcounter{dlibc}{0} % set initial value
	\renewcommand{\thedlibc}{\arabic{dlibc}} % set references to \roman
	\newcommand{\dlref}[1]{\ref{#1}} % reference in brackets
	\DeclareRobustCommand{\dlitem}{\refstepcounter{dlibc}\thedlibc. }

% model short-name
	\newcommand{\jitmodel}{\texttt}


% tools
	\newcommand{\tool}{\texttt}

% software metrics
	\newcommand{\metric}{\texttt}
	\newcommand{\eqmetric}[1]{\text{\texttt{#1}}}

\lstdefinestyle{cdefect}{
  escapeinside={*@}{@*},
  belowcaptionskip=1\baselineskip,
  breaklines=true,
  frame=single,
  xleftmargin=\parindent,
  language=C,
  showstringspaces=false,
  basicstyle=\footnotesize\ttfamily,
  keywordstyle=\bfseries\color{green!40!black},
  commentstyle=\itshape\color{purple!40!black},
  identifierstyle=\color{blue},
  stringstyle=\color{orange},
  numbers=left,
  numbersep=5pt,
  tabsize=4,
}

\lstdefinelanguage{gitdiff}{
  basicstyle=\footnotesize\ttfamily,
	morecomment=[f][\bfseries\color{black}]{diff\ --git},
	morecomment=[f][\color{gray}]{@@},
	morecomment=[f][\color{green!40!black}]{+\	},
	morecomment=[f][\color{green!40!black}]{+\#},
	morecomment=[f][\color{green!40!black}]{+s},
	morecomment=[f][\color{red!40!black}]{-\	},
}

\lstdefinestyle{diff}{
  escapeinside={*@}{@*},
  belowcaptionskip=1\baselineskip,
  breaklines=true,
  frame=single,
  xleftmargin=\parindent,
  language=gitdiff,
  showstringspaces=false,
  basicstyle=\footnotesize\ttfamily,
  keywordstyle=\bfseries\color{green!40!black},
  commentstyle=\itshape\color{purple!40!black},
  identifierstyle=\color{black},
  stringstyle=\color{orange},
  numbers=left,
  numbersep=5pt,
  tabsize=4,
}

\lstdefinestyle{query}{
    numbersep=5pt,
    tabsize=4,
    showspaces=false,
    showstringspaces=false,
    basicstyle=\footnotesize\ttfamily,
}

\graphicspath{ {./reference/} }

% \let\oldparagraph\paragraph
% \renewcommand{\paragraph}[1]{\noindent\oldparagraph{#1}}


\begin{document}

\title{Summary of JIT-SDP Studies}
%%
%% The "author" command and its associated commands are used to define
%% the authors and their affiliations.
%% Of note is the shared affiliation of the first two authors, and the
%% "authornote" and "authornotemark" commands
%% used to denote shared contribution to the research.
\author{Yunhua Zhao}
\affiliation{%
        \department{Department of Computer Science}
        \institution{CUNY Graduate Center}
  \streetaddress{365 5th Avenue}
  \city{New York}
  \state{NY}
  \country{USA}
  \postcode{10016}
}
\email{yzhao5@gradcenter.cuny.edu}

\author{Kostadin Damevski}
\affiliation{%
        \department{Department of Computer Science}
        \institution{Virginia Commonwealth University}
  \streetaddress{401 West Main Street}
  \city{Richmond}
  \state{VA}
  \country{USA}
  \postcode{23284}
}
\email{damevski@acm.org}


\author{Hui Chen}
\affiliation{%
        \department{Department of Computer \& Information Science}
        \institution{CUNY Brooklyn College}
  \streetaddress{2900 Bedford Avenue}
  \city{Brooklyn}
  \state{NY}
  \country{USA}
  \postcode{11210}
}
\email{huichen@acm.org}
\orcid{0000-0002-9840-4876}
\authornote{The corresponding author}
\additionalaffiliation{%
        \institution{CUNY Graduate Center}
        \department{Department of Computer Science}
  \streetaddress{365 5th Avenue}
  \city{New York}
  \state{NY}
  \country{USA}
  \postcode{10016}
}

%%
%% By default, the full list of authors will be used in the page
%% headers. Often, this list is too long, and will overlap
%% other information printed in the page headers. This command allows
%% the author to define a more concise list
%% of authors' names for this purpose.
\renewcommand{\shortauthors}{Zhao, Damevski, and Chen}




\maketitle





\newcounter{numjitpapers}
\begin{longtable}[c]{>{\refstepcounter{numjitpapers}}p{1.0in} p{0.25in} p{1in} p{2.65in}}
	\caption{JIT-SDP Researches and Primary Topics.\label{tab:jitsdp:topics}}\\
	\toprule
	\multicolumn{4}{c}{Begin of Table}\\
	\midrule
	Research
	& Year$^a$%\footnote{
	%				The publication year is from the online publication date if available. The
	%				online publication date may be different from the bibliographic or the final
	%				publication date.
	%}
	& Topic Area & Primary Topics \\
	\midrule
	\endfirsthead
	
	\midrule
	\multicolumn{4}{c}{Continuation of Table~\ref{tab:jitsdp:topics}}\\
	\midrule
	Research
	& Year$^a$%\footnote{The publication year is from the online publication date if available. The
	%online publication date may be different from the bibliographic or the final
	%publication date.}
	& Topic Area & Primary Topics \\
	\midrule
	\endhead
	
	\midrule
	\multicolumn{4}{p{5.1in}}{$^a${\footnotesize{The publication year is from the online publication date if available. The
				online publication date may be different from the bibliographic or the final
				publication date.}}}\\
	\endfoot
	
	\midrule
	\multicolumn{4}{c}{End of Table}\\
	\bottomrule
	\multicolumn{4}{p{5.1in}}{$^a${\footnotesize{The publication year is from the online publication date if available. The
				online publication date may be different from the bibliographic or the final
				publication date.}}}\\
	\endlastfoot
	
	\setcounter{numjitpapers}{1}
	Duan et al.~\cite{duan2021impact}
	& 2021
	& Noise reduction
	& impact of duplicate changes, i.e., identical changes applied to
	multiple SCM branches on prediction performance
	%The counter is~\thenumjitpapers
	\\ % replication=y
	
	Zhao et al.~\cite{zhao2021simplified}
	& 2021
	& Within-project model, application domain
	& assessment of a custom deep forest model for Android mobile apps
	\\ % replication=n
	
	Hoang et al.~\cite{hoang2020cc2vec}
	& 2020
	& Model, feature representation
	& Building a convolutional network to extract feature representations of
	software changes considering change structure and to use the features
	for defect prediction
	\\ % replication=https://github.com/CC2Vec/CC2Vec
	
	Kang et al.~\cite{kang2020predicting}
	& 2020
	& SDLC, application domain
	& within and cross-project SDP comparison and cost-benefit analysis
	for post-release changes in
	maritime software
	\\ % replication=n
	
	Tabassum et al.~\cite{tabassum2020investigation}
	& 2020
	& Cross-project model
	& assessment of cross-project SDP with an online
	learning (data stream learning) model
	\\ % replication=n
	
	Tian et al.~\cite{tian2020well}
	& 2020
	& Within-project model evaluation, SDLC
	& evaluations of long-term JIT-SDP for reliability improvement
	in terms of the usage-weighted defects and short-term JIT-SDP for
	early defect prediction
	\\ % replication=n
	
	Trautsch et al.~\cite{trautsch2020static}
	& 2020
	& Sub-change prediction, metrics, SZZ
	& investigation of extensive list of metrics including static analysis
	warnings and improved SZZ algorithm for sub-change (files in a change)
	defect prediction
	\\ % replication=https://zenodo.org/record/3974204#.YHIprj8pBEY
	
	Bennin et al.~\cite{bennin2020revisiting}
	& 2020
	& Data distribution
	& investigation of concept drift within software projects
	\\ % replication=n
	
	Zhu et al.~\cite{zhu2020within}
	& 2020
	& Model
	& investigation of effectiveness of convolutional neural networks
	on defect prediction using software metrics as input features
	\\ % replication=n
	
	Yan et al.~\cite{yan2020effort}
	& 2020
	& Prediction setting, application domain
	& investigation of the effectiveness of supervised (\jitmodel{CBS+},
	\jitmodel{CBS}, \jitmodel{OneWay}, and \jitmodel{EALR}) and unsupervised
	(\jitmodel{LT} and \jitmodel{Code Churn}) {\em effort-aware} JIT-SDP in
	an {\em industry} setting (on Alibaba projects)
	\\ % replication=n
	
	Khanan et al.~\cite{khanan2020jitbot}
	& 2020
	& Prediction setting, applications of JIT-SDP
	& design of explainable JIT-SDP bot that ``explains'' a defect
	prone change with the ``contribution'' of software metrics to
	the defect proneness
	\\ % replication=n
	
	Li et al.~\cite{li2020effort}
	& 2020
	& Model
	& investigation of semi-supervised effort-aware JIT-SDP
	using a tri-training method (also see Zhang et al.~\cite{zhang2019effort})
	\\ % replication=https://github.com/NJUST-IDAM/EATT
	% conference version of Zhang et al. (2019)
	
	Catolino et al.~\cite{catolino2019cross}
	& 2019
	& Application domain, model
	& investigation of cross-project JIT-SDP in {\em mobile platforms}
	and comparison of four classifiers and four ensemble techniques.
	\\ %replication=n
	
	Kondo et al.~\cite{kondo2020impact}
	& 2019
	& Software metrics
	& design and investigation of ``context metrics'', a metric
	measure the complexity or the size of the surrounding lines
	of a change.
	\\ % replication=n
	
	Fan et al.~\cite{fan2019impact}
	& 2019
	& Noise reduction
	& investigation of impact of SZZ algorithms and labeling errors
	\\ % replication=https://github.com/YuanruiZJU/SZZ-TSE
	
	Hoang et al.~\cite{hoang2019deepjit}
	& 2019
	& Model
	& investigation of cross-validation, short-term, and long-term prediction
	of convolutional neural networks using tokens (words) from both commit
	messages and changes as input.
	\\ % replication=https://github.com/AnonymousAccountConf/
	
	Pascarella et al.~\cite{pascarella2019fine}
	& 2019
	& Prediction setting
	& Predicting whether in the specific files, contained in
	a commit, that are defect-inducing
	\\ %replication=y (claimed), cannot locate it
	
	Huang et al.~\cite{huang2019revisiting}
	& 2018
	& Model
	& investigation of a supervised effort-aware model (called
	\jitmodel{CBS+}) combining Kamei et al.'s supervised \jitmodel{EALR}
	model~\cite{kamei2012large} and Yang et al.'s unsupervised
	\jitmodel{LT}~\cite{yang2016effort}
	\\ % replication=https://zenodo.org/record/1432582#.W6YyU2gzaUl
	
	Cabral et al.~\cite{cabral2019class}
	& 2019
	& Model
	& investigation of an Oversampling Online Bagging (ORB) to tackle class
	imbalance evolution in an online JIT-SDP scenario while considering
	verification latency
	\\ %replication=http://doi.org/10.5281/zenodo.2555695
	
	
	Zhang et al.~\cite{zhang2019effort}
	& 2019
	& Model
	& investigation of semi-supervised effort-aware JIT-SDP
	using a tri-training method (also see Li et al.~\cite{li2020effort})
	\\ %replication=https://github.com/NJUST-IDAM/EATT
	% journal version of Li et al. (2020)
	
	
	Guo et al.~\cite{guo2018bridging}
	& 2018
	& Evaluation
	& investigation of the relationship between classification performance
	and the cost-effectiveness performance metrics to obtain insights, e.g.,
	that there is great variability in repair effort.
	\\ %replication=https://github.com/yuchen1990/EAposter
	
	Young et al.~\cite{young2018replication}
	& 2018
	& Model
	& comparison of the prediction of defect-prone changes using traditional
	machine learning techniques and ensemble learning
	algorithms
	\\ %replication=n
	
	Chen et al.~\cite{chen2018multi}
	& 2017
	& Model
	& formulating prediction as a dual-objective optimization
	problem based on logistic regression and NSGA-II to balance
	the benefit, i.e., the predicted defects and the cost, i.e., the
	review efforts.
	\\ %replication=https://github.com/Hecoz/Multi-Project-Learning
	
	
	Fu and Menzies~\cite{fu2017revisiting}
	& 2017
	& Model
	& investigation of Yang et al.'s unsuperivsed models (\jitmodel{LT})
	and the prosed \jitmodel{OneWay} that uses the supervise models to
	prune unsupervised models
	\\ % replication=https://github.com/WeiFoo/RevisitUnsupervised
	% snowball
	
	Huang et al.~\cite{huang2017supervised}
	& 2017
	& Model
	& investigation of a suprervised effort-aware model (called \jitmodel{CBS})
	combining Kamei et al.'s supervised \jitmodel{EALR}
	model~\cite{kamei2012large} and Yang et al.'s unsupervised
	\jitmodel{LT}~\cite{yang2016effort}
	\\ % replication=https://zenodo.org/record/836352#.WX2MlYiGOUk
	
	Liu et al.~\cite{liu2017code}
	& 2017
	& Model
	& investigation of the effectiveness of code churn based unsupervised
	defect prediction model (\jitmodel{CCUM}) for effort-aware prediction
	\\ % replication=n
	
	McIntosh and Kamei~\cite{mcintosh2017fix}
	& 2017
	& Prediction setting
	& investigation of evolving nature of software project with software
	projects with insights, such as, JIT models that should be retrained
	using recently recorded data
	\\ %replication=https://github.com/software-rebels/JITMovingTarget
	
	
	
	Yang et al.~\cite{yang2017tlel},
	& 2017
	& Model
	& investigation of a two-layer ensemble model (\jitmodel{TLEL}) for effort-aware
	prediction
	\\ % replication=n
	
	Kamei et al.~\cite{kamei2016studying}
	& 2016
	& Prediction setting, model
	& Examination of an ensemble approach for cross-project JIT-SDP with
	random forest base learner (also see Fukushima et
	al.~\cite{fukushima2014empirical})
	\\ % replication=n
	
	Tourani and Adams~\cite{tourani2016impact}
	& 2016
	& Data, metrics
	& investigation of using issue and review discussions to predict the
	defect-proneness of software patchs
	\\ %replication=n
	
	Herzig et al.~\cite{herzig2016impact}
	& 2016
	& Data, model
	& investigation of {\em tangled changes} on defect prediction and
	examination of a multi-predictor approach to untangle changes
	\\ % replication=n
	
	Yang et al.~\cite{yang2016effort}
	& 2016
	& Model
	& investigation of the predictive power of simple unsupervised models,
	such as, \jitmodel{LT} and \jitmodel{AGE} in effort-aware JIT defect
	prediction and comparison of a variety of supervised and unsupervised
	models
	\\ % replication=http://ise.nju.edu.cn/yangyibiao/jit.html # dead link
	
	Rosen et al.~\cite{rosen2015commit}
	& 2015
	& Tool
	& describing a publicly available defect prediction tool called Commit
	Guru
	\\ % replication=n but link yes commit.guru
	
	Tan et al.~\cite{tan2015online}
	& 2015
	& Model
	& proposing an online JIT-SDP model and investigating class imbalance
	problem and time-sensitive change classification for defect prediction
	where bag-of-words feature of commit message, static code metrics, the
	node type in abstract syntax trees and meta data features.
	\\ % replication=n
	
	Yang et al.~\cite{yang2015deep}
	& 2015
	& Model
	& proposing a model called \jitmodel{Deeper} consisting of a deep belief
	network and a logistic regression classifier to predict defect proneness
	of software changes
	\\ % replication=n
	
	% Wiese et al.~\cite{wiese2015empirical}
	%  	& 2015
	%		&
	%		&
	% \\
	
	Fukushima et al.~\cite{fukushima2014empirical}
	& 2014
	& Model
	& Examination of an ensemble approach for cross-project JIT-SDP with
	random forest base learner (also see Kamei et
	al.~\cite{kamei2016studying})
	\\ % replication=n
	
	Jiang et al.~\cite{jiang2013personalized}
	& 2013
	& Model
	& Building (file) change-level defect prediction model for each developer from
	file modification histories (i.e., a personlized defect prediction)
	\\ % replication=n
	
	Singh and Chaturvedi~\cite{singh2013improving}
	& 2013
	& Metrics, model
	& investigation of entropy change metrics, metrics decay (aging)
	function, and defect prediction model using linear regression and support
	vector machine for defect prediction
	\\ % replication=n
	
	% Tass{\'e}~\cite{tasse2013using},
	%	& 2013
	%	& Prediction setting, model
	%	& predicting whether the defect proneness of a given file in a short term
	%	(e.g., next 3 months) from analyzing the number of changes and their
	%	types of the change bursts
	%	\\
	
	Kamei et al.~\cite{kamei2012large}
	& 2012
	& Model
	& Predicting defect-proneness of software changes with logistic regression
	and quality assurance effort of software changes with linear regression  (\jitmodel{EALR})
	from software metrics
	\\ % replication=http://research.cs.queensu.ca/~kamei/jittse/jit.zip
	
	Kim et al.~\cite{kim2008classifying}
	& 2008
	& Model
	& Predicting with a Support Vector Machine (\jitmodel{SVM})
	using the {\em bag-of-words} features of the identifiers in added and
	deleted source code and the words in {\em file} change logs to classify
	changes as being defect-inducing or clean
	\\ % replication=n
	
	Mockus and Weiss~\cite{mockus2000predicting}
	\label{numjitpapers}
	\newcounter{numdljitpapers}
	\newcounter{numsbjitpapers}
	\setcounter{numsbjitpapers}{4}%5
	\refstepcounter{numsbjitpapers}\thenumsbjitpapers\label{numsbjitpapers}
	\setcounter{numdljitpapers}{\numexpr\value{numjitpapers}-\value{numsbjitpapers}-1}
	\refstepcounter{numdljitpapers}\thenumdljitpapers\label{numdljitpapers}
	& 2000
	& Prediction setting, model
	& Predicting from software change metrics with \jitmodel{logistic
		regression} the defect-proneness of the Initial Modification Requests
	(IMR) in 5ESS network switch project where IMRs may consist of multiple
	Modification Requests (MR) corresponding to multiple changes
	%The counter is~\thenumjitpapers
	\\ % replication=n
\end{longtable}






\bibliographystyle{ACM-Reference-Format}
\bibliography{
	jitsdp,jitsdp_snowball}


\end{document}
